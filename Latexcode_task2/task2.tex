\documentclass[12pt]{article}
\usepackage[utf8]{inputenc}
\usepackage{caption}
\usepackage{graphicx}
\usepackage{amsmath}
\usepackage{fancyhdr}
\usepackage{adjustbox}
\usepackage[most]{tcolorbox}
\pagestyle{fancy}
\usepackage{color}
\usepackage{titlesec}
\usepackage{tikz}
\usepackage{wrapfig} 
\usepackage[a4paper,top=1.8cm,bottom=0.8cm,left=2.5cm,right=2.5cm,margin=1in]{geometry}
\usepackage{xcolor}
\usepackage{enumitem}
\usepackage{setspace}
\usepackage{titling}
\usepackage{lipsum}
\usepackage{float}
\usepackage{subcaption}
\pagestyle{empty}
\setlength{\parindent}{0pt}
\setlength{\headheight}{61.25537pt}
\addtolength{\topmargin}{-49.25537pt}

\setlength{\parskip}{0.8em}
\setstretch{1}
\usepackage{xcolor} % For colored text
\fancyhf{} % clear
\definecolor{chapterblue}{RGB}{0,174,239} 
\definecolor{darkskyblue}{rgb}{0.0, 0.5, 1.0}
\definecolor{skyblue}{RGB}{135, 206, 235}
\usepackage{newtxtext,newtxmath}
\titleformat{\section}
{\fontsize{67}{48}\selectfont\bfseries\color{chapterblue}}{\thesection}{1em}{}
% Title format
\titleformat{\section}
{\normalfont\Large\bfseries\color{chapterblue}}{\thesection}{1em}{}
% Header/Footer settings
\pagestyle{fancy}
\cfoot{2019-20}
\renewcommand{\headrulewidth}{0pt}
\renewcommand{\footrulewidth}{0pt}
\begin{document}
\pagestyle{empty}
\thispagestyle{fancy}
\renewcommand{\headrulewidth}{0pt} 
\fancyhead[R]{
Name: B SHAMALA DEVI\\
Batch: COMETFWC030\\
Date: 12-june -2025
}


\vspace{-1.9em}
\begin{tikzpicture}
\vspace{-2em}
\end{tikzpicture}
\vspace{-0.5em}
\begin{center}
\vspace{-2em}
{\textcolor{chapterblue}{\hspace{10em}
\textbf{

}
\end{center}
\vspace{-2em}
\noindent
\begin{tikzpicture}
\draw[chapterblue, line width=4pt] (0,0) -- (\linewidth,0);
\end{tikzpicture}
\vspace{-1em}

\begin{enumerate}
\setcounter{enumi}{3} % Start numbering from (iv)
\item The number of zeroes is 1. (Why?)
\item The number of zeroes is 1. (Why?)
\item The number of zeroes is 4. (Why?)
\end{enumerate}
\vspace{-1.5em}
\begin{center}
\noindent\textcolor{cyan}{\textbf{EXERCISE 2.1}}
\end{center}
1. The graphs of \( y = p(x) \) are given in Fig.~2.10 below, for some polynomials \( p(x) \).Find the number of zeroes of \( p(x) \), in each case

\includegraphics[width=16cm,height=8cm]{graph.jpeg}
\begin{center}
\textcolor{cyan}{\textbf{fig. 2.10}}
\end{center}
\section*{2.3 Relationship between Zeroes and Coefficients of a Polynomial}
You have already seen that zero of a linear polynomial \( ax + b \) is \( -\frac{b}{a} \).We will now try
\\to answer the question raised in Section 2.1 regarding the relationship between zeroes and coefficients of a q7adratic polynomial.For this, let us take a quadratic polynomial,
\\ say\(p(x) = 2x^2 - 8x + 6.\)In Class IX, you have learnt how to factorise quadratic polynomials by splitting the middle term.So, here we need to split the middle term 
\\ \(-8x\)’ as a sum of two terms,whose product is \( 6 \times 2x^2 = 12x^2 \).So, we write:
\begin{align*}2x^2 - 8x + 6 &= 2x^2 - 6x - 2x + 6 &= 2x(x - 3) - 2(x - 3) \\
&= (2x - 2)(x - 3)&= 2(x - 1)(x - 3)
\end{align*}

\newpage

\begin{tikzpicture}
\draw[chapterblue, line width=4pt] (0,0) -- (\linewidth,0);
\end{tikzpicture}
\vspace{-1em}


\begin{center}
\noindent\textcolor{cyan}{\textbf{EXERCISE 2.2}}
\end{center}
\textbf{1.} Find the zeroes of the following quadratic polynomials and verify the relationship between the zeroes and the coefficients:  
\begin{itemize}
  (i) \(x^2 - 2x - 8\ \hspace{2em}(ii) \(4s^2 - 4s + 1\ \hspace{2em} (iii) \(6x^2 - 3 - 7x\
  \\(iv) \(4u^2 + 8u\ \hspace{2.5em} (v) \(t^2 - 15\ \hspace{2.8em} (vi) \(3x^2 - x - 4\
\end{itemize}

\textbf{2.} Find a quadratic polynomial each with the given numbers as the sum and product of its zeroes respectively.  \begin{itemize}
   (i) (\ \frac{1}{4}, -1 \)\hspace{2em},(ii) (\ \sqrt{2}, \frac{1}{3})\ \hspace{2em}(iii) (\0, \sqrt{5})\
  \\ (iv) (\1, 1)\ \hspace{2em}(v) (\ -\frac{1}{4}, \frac{1}{4})\ \hspace{2em}(vi) (\4, 1)\
\end{itemize}


\vspace{1em}
\textcolor{cyan}{\textbf{2.4 Division Algorithm for Polynomials}} \\
You know that a cubic polynomial has at most three zeroes. However, if you are given 

\ only one zero, can you find the other two? For this, let us consider the cubic polynomial
\ $\[x^3 - 3x^2 - x + 3\]\$.  If we tell you that one of its zeroes is 1, then you know that \(x - 1\) is 
\ a factor of \(x^3 - 3x^2 - x + 3\). So, you can divide x\(3 - 3x^2 - x + 3\) by \(x - 1\), as you have 
\ learnt in Class IX, to get the quotient \(x^2 - 2x - 3\).  

\hspace{2em} Next, you could get the factors of \(x^2 - 2x - 3\) , by splitting the middle term, as  
\\ (x + 1)(x - 3). \ This would give you 
\begin{center}
 x^3 - 3x^2 - x + 3 = (x - 1)(x^2 - 2x - 3)
\)
\\ \(
= (x - 1)(x + 1)(x - 3)
\)   
\end{center}

\hspace{2em}  So , \ all the three zeroes of the cubic polynomial are now known to you as 
\\ 1, -1, 3.  

\hspace{2em}Let us discuss the method of dividing one polynomial by another in some detail. Before noting the steps formally, consider an example.  

\vspace{1em}
\begin{flushright}
\vspace{-3em}\includegraphics[width=5.5cm,height=6cm]{div.jpg} 
\end{flushright}
\vspace{-13em}
\textcolor{cyan}{\textbf{Example 6:}} Divide \(2x^2+3x+1\) by \(x+2\).

\textcolor{cyan}{\textbf{Solution:}} Note that we stop the division process when 
\\ either the remainder is zero or its degree is less than 
\\ the degree of the divisor. So, here the quotient is 2x - 1 and
\\  the remainder is 3 . Also,  \\
\hspace{1.8em} \((2x - 1)(x + 2) + 3 = 2x^2 + 3x - 2 + 3 = 2x^2 + 3x + 1\)
\\  i.e., \hspace{0.5em} \(2x^2 + 3x + 1 = (x + 2)(2x - 1) + 3\)

Therefore, Dividend = Divisor × Quotient + Remainder  

Let us now extend this process to divide a polynomial by a quadratic polynomial.
\newpage
\begin{tikzpicture}
\draw[chapterblue, line width=4pt] (0,0) -- (\linewidth,0);
\end{tikzpicture}
\vspace{-1em}


\textbf{So, } \(2x^4 - 3x^3 - 3x^2 + 6x - 2 = (x^2 - 2)(2x^2 - 3x + 1).\)
\vspace{1em} \\ Now, by splitting $-3x$, we factorise $2x^2 - 3x + 1$ as $(2x - 1)(x - 1)$. So, its zeroes 
\vspace{1em} are given by $x = \frac{1}{2}$ and $x = 1$. Therefore, the zeroes of the given polynomial are 
\vspace{1em} \\ (\sqrt{2}, \quad -\sqrt{2}, \quad \frac{1}{2}, \quad \text{and } 1.\)

\begin{center}
\noindent\textcolor{cyan}{\textbf{EXERCISE 2.3}}
\end{center}

\begin{enumerate}
\item Divide the polynomial $p(x)$ by the polynomial $g(x)$ and find the quotient and remainder in each of the following:

\begin{enumerate}
    \item $p(x) = x^3 - 3x^2 + 5x - 3$, \quad $g(x) = x^2 - 2$
    \item $p(x) = x^4 - 3x^2 + 4x + 5$, \quad $g(x) = x^2 + 1 - x$
    \item $p(x) = x^4 - 5x + 6$,        \quad $g(x) = 2 - x$
\end{enumerate}

\item Check whether the first polynomial is a factor of the second polynomial by dividing the second polynomial by the first polynomial:
\begin{enumerate}
    \item $t^2 - 3$, \quad $2t^4 + 3t^2 - 2t^2 - 9t - 12$
    \item $x^2 + 3x + 1$, \quad $3x^4 + 5x^3 - 7x^2 + 2x + 2$
    \item $x^3 - 3x + 1$, \quad $x^5 - 4x^3 + x^2 + 3x + 1$
\end{enumerate}

\item Obtain all other zeroes of $3x^4 + 6x^3 - 2x^2 - 10x - 5$, if two of its zeroes are $\frac{\sqrt{5}}{3}$ and $-\frac{\sqrt{5}}{3}$.

\item On dividing $x^3 - \left(3x^2 + x + 2\right)$ by a polynomial $g(x)$, the quotient and remainder were $x - 2$     and $-2x + 4$, respectively. Find $g(x)$.

\item Give examples of polynomials $p(x)$, $g(x)$, $q(x)$ and $r(x)$ which satisfy the division algorithm and
\begin{enumerate}
    \item $\deg p(x) = \deg q(x)$ \hspace{1em}  (ii)$\deg q(x) = \deg r(x)$  \hspace{1em}(iii) $\deg r(x) = 0$
\end{enumerate}


\begin{center}
\noindent\textcolor{cyan}{\textbf{EXERCISE 2.4 (Optional*)}}
\end{center}


\begin{enumerate}
\hspace{-3em}1.Verify that the numbers given alongside of the cubic polynomials below are their \hspace{-1.5em} zeroes
\hspace{-0em}Also verify the relationship between the zeroes and the coefficients in each case:
\begin{enumerate}
\item $2x^3 + x^2 - 5x + 2$; ;\hspace{-0.5em}\quad $\frac{1}{2}, 1, -2$    \hspace{3em} (ii)$x^3 - 4x^2 + 5x - 2$;\hspace{-0.5em}\quad $2, 1, 1$
\end{enumerate}

\hspace{-3em} 2. Find a cubic polynomial with the sum, sum of the product of its zeroes 
\hspace{-0em} taken two at a time, and the product of its zeroes as $2, -7, -14$ respectively.
\end{enumerate}

\vspace{1em}
\begin{tikzpicture}
\draw[chapterblue] (0,0) -- (10,0); % A thick blue line from (0,0) to (4,0)
\end{tikzpicture}
\begin{itemize}[label={\textcolor{cyan}{*}}]
\item  Theseexercises are not from point of view
\end{document}
