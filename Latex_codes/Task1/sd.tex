\documentclass[12pt]{article}
\usepackage[utf8]{inputenc}
\usepackage{caption}
\usepackage{graphicx}
\usepackage{amsmath}
\usepackage{fancyhdr}
\usepackage{adjustbox}
\usepackage[most]{tcolorbox}
\pagestyle{fancy}
\usepackage{color}
\usepackage{titlesec}
\usepackage{tikz}
\usepackage{wrapfig}  
\usepackage[a4paper,top=1.8cm,bottom=0.8cm,left=2.5cm,right=2.5cm,margin=1in]{geometry}
\usepackage{xcolor}
\usepackage{enumitem} % <-- This allows item customization
\usepackage{setspace}
\usepackage{titling}
\usepackage{lipsum}
\usepackage{float}
\usepackage{subcaption}
\usepackage{newtxtext,newtxmath}
\pagestyle{empty}
\setlength{\parindent}{0pt}
\setlength{\headheight}{61.25537pt}
\addtolength{\topmargin}{-49.25537pt}
\setlength{\parskip}{0.8em}
\setstretch{1}

\definecolor{chapterblue}{RGB}{0,174,239} 
\definecolor{darkskyblue}{rgb}{0.0, 0.5, 1.0}
\definecolor{skyblue}{RGB}{135, 206, 235}
\definecolor{myblue}{RGB}{0, 204, 255}

\titleformat{\section}
  {\normalfont\Large\bfseries\color{chapterblue}}{\thesection}{1em}{}

\fancyhf{}
\cfoot{2019-20}
\renewcommand{\headrulewidth}{0pt}
\renewcommand{\footrulewidth}{0pt}

\begin{document}
\pagestyle{empty}
\thispagestyle{fancy}
\renewcommand{\headrulewidth}{0pt} 
\fancyhead[L]{\includegraphics[width=5cm, height=2cm]{comet.jpg}}
\fancyhead[R]{Name: B SHAMALA DEVI\\Batch: COMETFWC030\\Date: 21-May 2025}

\begin{center}
   \hspace{-7em} \includegraphics[width=100pt]{SCANNER.jpg} 
\end{center}

\vspace{-1.9em}
\begin{tikzpicture}
    \hspace{31em}\draw[line width=2pt, draw=cyan ] (0,0) rectangle (3,2); 
    \draw[line width=3pt, draw=cyan] (3,0)--(3,2);
    \node at (1.5,1) {\textcolor{black}{\fontsize{60pt}{60pt}\selectfont \textbf{2}}};
\end{tikzpicture}

\vspace{-0.5em}
\begin{center}
    \vspace{-2em}
    {\textcolor{chapterblue}{\hspace{10em}
        \textbf{
            {\fontsize{30}{40}\selectfont P}
            \fontsize{20}{34}\selectfont OLYNOMIALS
        }
    }}
\end{center}

\vspace{-2em}
\noindent
\begin{tikzpicture}
  \draw[chapterblue, line width=4pt] (0,0) -- (\linewidth,0);
\end{tikzpicture}

\vspace{-1em}
\section*{2.1 Introduction}
% (Your intro content continues...)

\vspace{-2em}
\section*{2.2 Geometrical Meaning of the Zeroes of a Polynomial}

You know that a real number $k$ is a zero (or root) of a polynomial $p(x)$ if $p(k) = 0$.

\begin{itemize}
  \item Consider a linear polynomial $ax + b$, where $a \ne 0$. The graph of $y = ax + b$ is a straight line.
  \item For example, the graph of $y = 2x + 3$ passes through $(-2, -1)$ and $(2, 7)$.
  \item From Fig.~2.1, the graph of $y = 2x + 3$ intersects the $x$-axis at $x = -\frac{3}{2}$.
  \item So, the zero of the polynomial is the $x$-coordinate of the point where the graph intersects the $x$-axis.
  \item In general, for $y = ax + b$, the graph intersects the $x$-axis at $\left( \frac{-b}{a}, 0 \right)$ and hence has exactly one zero.
  \item Now consider the quadratic polynomial $y = x^2 - 3x - 4$.
  \item Let us list a few values in Table 2.1 and examine the graph.
  \item It intersects the $x$-axis at two points.
  \item The number of zeroes of a polynomial equals the number of points where its graph intersects the $x$-axis.
\end{itemize}

\includegraphics[width=6cm,height=3cm]{table.jpg}
\begin{flushright}
    \vspace{-9em}\includegraphics[width=9cm,height=8cm]{line.jpg} 
\end{flushright}

\vspace{-13em}
\hspace{2em}From Fig.~2.1, you can see that the graph of $y = 2x + 3$ intersects the $x$-axis at $x = -\frac{3}{2}$, the zero of the polynomial.

\vspace{-5.7em}
\begin{center}
\textcolor{myblue}{\textbf{Fig. 2.10}}
\end{center}

\textbf{Note:} Plotting graphs of quadratic or cubic polynomials is not required for evaluation.

\begin{itemize}
  \item The number of zeroes is 1. (Why?)
  \item The number of zeroes is 1. (Why?)
  \item The number of zeroes is 4. (Why?)
\end{itemize}

\vspace{-1em}
\begin{center}
\textcolor{cyan}{\textbf{EXERCISE 2.1}}\\
The graphs of $y = p(x)$ are shown below for some polynomials. Find the number of zeroes in each case.
\end{center}

\includegraphics[width=15cm,height=5cm]{graph.jpeg}
\begin{center}
\textcolor{myblue}{\textbf{Fig. 2.10}}
\end{center}

\section*{2.3 Relationship between Zeroes and Coefficients of a Polynomial}

You have seen that the zero of a linear polynomial $ax + b$ is $-\frac{b}{a}$.

\begin{itemize}
  \item Consider the quadratic polynomial $p(x) = 2x^2 - 8x + 6$.
  \item Split the middle term $-8x$ into two terms whose product is $2 \times 6 = 12$. So, $-8x = -6x - 2x$.
  \item Factorising:
\end{itemize}

\begin{align*}
2x^2 - 8x + 6 &= 2x^2 - 6x - 2x + 6 \\
              &= 2x(x - 3) - 2(x - 3) \\
              &= (2x - 2)(x - 3) \\
              &= 2(x - 1)(x - 3)
\end{align*}

\begin{itemize}
  \item So, the zeroes of the quadratic polynomial are $x = 1$ and $x = 3$, and they are related to the coefficients.
\end{itemize}

\end{document}
