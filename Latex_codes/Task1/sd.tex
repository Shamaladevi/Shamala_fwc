\documentclass[12pt]{article}
\usepackage[utf8]{inputenc}
\usepackage{caption}
\usepackage{graphicx}
\usepackage{amsmath}
\usepackage{fancyhdr}
\usepackage{adjustbox}
\usepackage[most]{tcolorbox}
\pagestyle{fancy}
\usepackage{color}
\usepackage{titlesec}
\usepackage{tikz}
\usepackage{wrapfig}  
\usepackage[a4paper,top=1.8cm,bottom=0.8cm,left=2.5cm,right=2.5cm,margin=1in]{geometry}
\usepackage{xcolor}
\usepackage{enumitem}
\usepackage{setspace}
\usepackage{titling}
\usepackage{lipsum}
\usepackage{float}
\usepackage{subcaption}
\pagestyle{empty}
\setlength{\parindent}{0pt}
\setlength{\headheight}{61.25537pt}
\addtolength{\topmargin}{-49.25537pt}

\setlength{\parskip}{0.8em}
\setstretch{1}
\usepackage{xcolor}      % For colored text
\fancyhf{} % clear
\definecolor{chapterblue}{RGB}{0,174,239} 
\definecolor{darkskyblue}{rgb}{0.0, 0.5, 1.0}
\definecolor{skyblue}{RGB}{135, 206, 235}
\usepackage{newtxtext,newtxmath}
\titleformat{\section}
  {\fontsize{67}{48}\selectfont\bfseries\color{chapterblue}}{\thesection}{1em}{}
% Title format
\titleformat{\section}
  {\normalfont\Large\bfseries\color{chapterblue}}{\thesection}{1em}{}
% Header/Footer settings
\pagestyle{fancy}
\cfoot{2019-20}
\renewcommand{\headrulewidth}{0pt}
\renewcommand{\footrulewidth}{0pt}
\begin{document}
\pagestyle{empty}
\thispagestyle{fancy}
\renewcommand{\headrulewidth}{0pt} 
\fancyhead[L]{
        \includegraphics[width=5cm, height=2cm]{comet.jpg} 
        }
\fancyhead[R]{
    Name: B SHAMALA DEVI\\
    Batch: COMETFWC030\\
    Date: 21-may 2025
}
\begin{center}
   \hspace{-7em} \includegraphics[width=100pt]{SCANNER.jpg} 
\end{center}

\hfill 

\vspace{-1.9em}
\begin{tikzpicture}
    \vspace{-2em}\hspace{31em}\draw[line width=2pt, draw=cyan ] (0,0) rectangle (3,2); \draw[line width=3pt, draw=cyan] (3,0)--(3,2);
     \node at (1.5,1) {\textcolor{black}{\fontsize{60pt}{60pt}\selectfont \textbf{2}}};
\end{tikzpicture}
 \vspace{-0.5em}
\begin{center}
\vspace{-2em}
    {\textcolor{chapterblue}{\hspace{10em}
        \textbf{
{\fontsize{30}{40}\selectfont P}
\fontsize{20}{34}\selectfont OLYNOMIALS }   }
                       }
\end{center}
\vspace{-2em}
\noindent
\begin{tikzpicture}
  \draw[chapterblue, line width=4pt] (0,0) -- (\linewidth,0);
\end{tikzpicture}
\vspace{-1em}
\section*{2.1 Introduction}
In Class IX, you have studied polynomials in one variable and their degrees. Recall that if \( p(x) \) is a polynomial in \( x \), the highest power of \( x \) in \( p(x) \) is called the \textbf{degree of the polynomial} \( p(x) \). For example, \( 4x + 2 \) is a polynomial in the variable \( x \) of degree 1, \( 2y^2 - 3y + 4 \) is a polynomial in the variable \( y \) of degree 2, 
\( 5x^3 - 4x^2 + x - \sqrt{2} \) 
\begin{spacing}{2}
is a polynomial in the variable \( x \) of degree 3 and \( 7u^6 - \frac{3}{2}u^4 + 4u^2 + u - 8 \) is a polynomial

in the variable \( u \) of degree 6. Expressions like \( \frac{1}{x-1} \), \( \sqrt{x+2} \), \( \frac{1}{x^2 + 2x + 3} \), etc., are not polynomials.

\hspace{2em} A polynomial of degree 1 is called a \textbf{linear polynomial}. For example, 
\( 2x - 3 \), \( \sqrt{3}x + 5 \), \( y + \sqrt{2} \), \( x - \frac{2}{11} \), $z + 4\frac{2}{3} + 1 ,$ etc., are all linear polynomials. Polynomials \\such as \( 2x + 5 - x^2 \), \( x^3 + 1 \), etc., are not linear polynomials. \\
\vspace{-0.2em}
\hspace{2em}
A polynomial of degree 2 is called a \textbf{quadratic polynomial}. The name ‘quadratic’
\\has been derived from the word ‘quadrate’, which means ‘square’. 
\( 2x^2 + 3x - \frac{2}{5} \) \\
\vspace{0.5em}
\( y^2 - 2 \), \( 2 - x^2 + \sqrt{3}x \), 
\( \frac{u}{3} - 2u^2 + 5 \), \( \sqrt{5}v^2 - \frac{2}{3}v \), 
\( 4z^2 + \frac{1}{7} \) are some examples of 
\\quadratic polynomials (whose coefficients are real numbers).More generally, any quadratic polynomial in \( x \) is of the form \( ax^2 + bx + c \), where \( a, b, c \) are real numbers and \( a \neq 0 \).A polynomial of degree 3 is called a \textbf{cubic polynomial}. Some examples of 
% Define custom color
\definecolor{myblue}{RGB}{0, 204, 255}

\begin{quote}
    
a cubic polynomial are \( 2 - x^3 \), \( x^3 \), \( \sqrt{2}x^3 \), \( 3 - x^2 + x^3 \), \( 3x^3 - 2x^2 + x - 1 \). In fact,the most general form of a cubic polynomial is
\end{quote}
\end{spacing}
\vspace{-3em}
\[ax^3 + bx^2 + cx + d,\]
where \( a, b, c, d \) are real numbers and \( a \neq 0 \).


\qquad Now consider the polynomial \( p(x) = x^2 - 3x - 4 \). Then, putting \( x = 2 \) in the polynomial, we get \( p(2) = 2^2 - 3 \cdot 2 - 4 = -6 \). The value ‘–6’, obtained by replacing \( x \) by 2 in \( x^2 - 3x - 4 \), is the value of \( x^2 - 3x - 4 \) at \( x = 2 \). Similarly, \( p(0) \) is the value of \( p(x) \) at \( x = 0 \), which is –4.
\\
\qquad If \( p(x) \) is a polynomial in \( x \), and if \( k \) is any real number, then the value obtained by replacing \( x \) by \( k \) in \( p(x) \), is called \textbf{the value of} \( p(x) \) \textbf{at} \( x = k \), and is denoted by \( p(k) \).

\qquad What is the value of \( p(x) = x^2 - 3x - 4 \) at \( x = -1 \)? We have :
\vspace{-1em}
\[p(-1) = (-1)^2 - \{3 \times (-1)\} - 4 = 0 \]

Also, note that $p (4) = 4^2 - (3 x 4) - 4 = 0.$ \\

\qquad As \( p(-1) = 0 \) and \( p(4) = 0 \), –1 and 4 are called the \textbf{zeroes} of the quadratic polynomial \( x^2 - 3x - 4 \). More generally, a real number \( k \) is said to be a \textbf{zero of a polynomial} \( p(x) \), if \( p(k) = 0 \).

\qquad You have already studied in Class IX how to find the zero of a linear polynomial.  
For example, if $k$ is a zero of $p(x) = 2x + 3$, then  
\ p(k) = 0  \  gives us 
\\ 2k + 3 = 0, \ i.e., \( k = \frac{3}{2} \) 
\\

\qquad In general, if \( k \) is a zero of \( p(x) = ax + b \), then 

\[p(k) = ak + b = 0 \Rightarrow ak = -b \Rightarrow k = -\frac{b}{a}.\]
So, the zero of the linear polynomial \( ax + b \) is \(  -\frac{b}{a} \) =
\( -\frac{\text{Constant term}} {\text{Coefficient of } x}\)


\noindent
Thus, the zero of a linear polynomial is related to its coefficients. Does this
happen in the case of other polynomials too? For example, are the zeroes of a \textit{quadratic}
polynomial also related to its coefficients?

\medskip

In this chapter, we will try to answer these questions. We will also study the
\ division algoirithm for polynomials

\vspace{-2em}

\section*{2.2 Geometrical Meaning of the Zeroes of a Polynomial}
\vspace{-1em}
\ You know that a real number \( k \) is a \emph{zero} (or \emph{root}) of a polynomial \( p(x) \) if \( p(k) = 0 \).  But why are these zeroes of a polynomial so important?To answer this, first we see the \textbf {geometrical} representations of linear and quadratic polynomials and the geometrical meaning of their zeroes.


Consider first a linear polynomial \( ax + b \), where \( a \ne 0 \).You have studied in Class IX that the graph of \( y = ax + b \) is a straight line. For example, the graph of \( y = 2x + 3 \) is a straight line passing through the points \( (-2, -1) \) and \( (2, 7) \).
% Define custom color
\definecolor{myblue}{RGB}{0, 204, 255}

% Header and footer setup
\pagestyle{fancy}
\fancyhf{}
% Left side of header - page number in blue
\fancyhead[L]{\textcolor{myblue}{\textbf{Polynomial}}}

% Right side of header - subject name
\fancyhead[R]{\textcolor{myblue}{\textsc{21}}}

% Blue line under header
\renewcommand{\headrulewidth}{1pt}
\renewcommand{\headrule}{\hbox to\headwidth{\color{myblue}\leaders\hrule height \headrulewidth\hfill}}

% Footer can be cleared or customized if needed
\fancyfoot[C]{}

% Increase row height

\includegraphics[width=6cm,height=3cm]{table.jpg}

\begin{flushright}
    \vspace{-9em}\includegraphics[width=9cm,height=8cm]{line.jpg} 
\end{flushright}
\vspace{-13em}
\hspace{2em}From Fig.~2.1, you can see\\
that the graph of \( y = 2x + 3 \)\\
intersects the \( x \)-axis mid-way\\
between \( x = -1 \) and \( x = -2 \),\\
\\ that is, at the point \( \left( -\frac{3}{2},\ 0 \right) \).\\
\\You also know that the zero of\\
\( 2x + 3 \) is \( -\frac{3}{2} \).
Thus, the zero of
\\the polynomial \( 2x + 3 \) is the\\
\( x \)-coordinate of the point where the\\
graph of \( y = 2x + 3 \) intersects the\\
\( x \)-axis.\\
\vspace{-5.7em}
\begin{center}
{\textcolor{myblue}{\textbf{Fig. 2.10}}}

\end{center}
\hspace{2em}In general, for a linear polynomial \( ax + b \), \( a \neq 0 \), the graph of \( y = ax + b \) is a\\

\text{straight line which intersects the x-axis at exactly one point, namely, } $\left( \frac{-b}{a},\ 0 \right).$
Therefore, the linear polynomial \( ax + b \), \( a \neq 0 \), has exactly one zero, namely, the\\
\( x \)-coordinate of the point where the graph of \( y = ax + b \) intersects the x-axis.\\

\hspace{2em}Now, let us look for the geometrical meaning of a zero of a quadratic polynomial.
\\Consider the quadratic polynomial \( x^2 - 3x - 4 \). Let us see what the graph* of\\
\( y = x^2 - 3x - 4 \) looks like.Let us list a few values of \( y = x^2 - 3x - 4 \) corresponding to\\
a few values for \( x \) as given in Table~2.1.\\
\begin{tikzpicture}
  \draw[chapterblue] (0,0) -- (7,0);  % A thick blue line from (0,0) to (4,0)
\end{tikzpicture}
\begin{itemize}[label={\textcolor{cyan}{*}}]
    \item Plotting of graphs of quadratic or cubic polynomials is not meant to be done by the students,nor is to be evaluated.
\end{itemize}
\begin{enumerate}
\setcounter{enumi}{3}  % Start numbering from (iv)
\item The number of zeroes is 1. (Why?)
\item The number of zeroes is 1. (Why?)
\item The number of zeroes is 4. (Why?)
\end{enumerate}
\vspace{-1.5em}
\begin{center}
\noindent\textcolor{cyan}{\textbf{EXERCISE 2.1}}
\end{center}
1. The graphs of \( y = p(x) \) are given in Fig.~2.10 below, for some polynomials \( p(x) \).Find the number of zeroes of \( p(x) \), in each case

\includegraphics[width=15cm,height=5cm]{graph.jpeg}
\begin{center}
\textcolor{myblue}{\textbf{Fig. 2.10}}
\end{center}
\section*{2.3 Relationship between Zeroes and Coefficients of a Polynomial}
You have already seen that zero of a linear polynomial \( ax + b \) is \( -\frac{b}{a} \).We will now try
\\to answer the question raised in Section 2.1 regarding the relationship between zeroes and coefficients of a quadratic polynomial.For this, let us take a quadratic polynomial,
\\ say\(p(x) = 2x^2 - 8x + 6.\)In Class IX, you have learnt how to factorise quadratic polynomials by splitting the middle term.So, here we need to split the middle term 
\\ \(-8x\)’ as a sum of two terms,whose product is \( 6 \times 2x^2 = 12x^2 \).So, we write:
\begin{align*}2x^2 - 8x + 6 &= 2x^2 - 6x - 2x + 6 &= 2x(x - 3) - 2(x - 3) \\
              &= (2x - 2)(x - 3)&= 2(x - 1)(x - 3)
\end{align*}


\end{document}
